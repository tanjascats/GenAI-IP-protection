
\begin{table*}
    \centering
\caption{Overview and terminology of categories of IP protection methods.}
\label{tab:terminology}

    \resizebox{\textwidth}{!}{\begin{tabular}{l|ll}
    \toprule
         Category & Definition & Synonymous terms \\
         \midrule
         Dataset sanitisation & Modifying training dataset before it is used for model training & - \\
         \hline
         Prompt modification & Modifying textual prompts in T2I scenario at inference time & - \\
         \hline
         Adversarial perturbations & Applying noise to train data samples intentionally crafted to & adversarial noise~\cite{zhao_unlearnable_2023}, style cloak/\\
 & disrupt the generation process&cloaking~\cite{shan_glaze_2023}, poisoning~\cite{liu_toward_2023},\\
 & &adversarial watermark(-ing)~\cite{zhang_editguard_2023}, \\
 & &watermark(-ing)\cite{ye_duaw_2023}, immunisation~\cite{salman_raising_2023}\\
        \hline
         Concept removal & Modifying the learning process to affect the downstream& concept unlearning~\cite{zhang_forget-me-not_2023}, concept \\
 & content generation&ablation~\cite{kumari_ablating_2023}, data redaction~\cite{kong_data_2023}, \\
 & &concept erasure~\cite{gandikota_erasing_2023}\\
        \hline
         Watermarking & The embedding of imperceptible signals into the content to & forensic watermarking~\cite{zhang_editguard_2023} \\
 & assert ownership or trace unauthorised use&\\
        \hline
         Analytical data attribution & Applying post-hoc analytical methods to identify the & - \\
 & contribution of specific train samples to the generated outputs&\\
        \hline
         Testing memorisation & Quantifying the memorisation capabilities of an underlying& -\\
 & GAI model&\\
 \bottomrule
    \end{tabular}}
    
    
\end{table*}